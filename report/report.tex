\documentclass[12pt]{article}

\newcommand{\reporttitle}{Crouzeix-Raviart Element for the Stokes Problem}
\newcommand{\accentcolor}{solarized-orange}
\newcommand{\urlcolor}{solarized-blue}

\usepackage{amsmath}
\usepackage{mathrsfs}
\usepackage{amsthm}
\usepackage{amsfonts}
\usepackage{bm}
\usepackage{amssymb}
\usepackage{stmaryrd}

% Sets.
\newcommand{\R}{\mathbb{R}}
\newcommand{\RT}{\mathbb{R}^2}
\newcommand{\N}{\mathbb{N}}

\newcommand{\PK}[1]{\mathbb{P}_{#1}}

\newcommand{\LT}{\mathscr{L}^2}
\newcommand{\HO}{\mathscr{H}^1}
\newcommand{\Tau}{\mathcal{T}}

% Vectors and operators.
\newcommand{\Vector}[1]{\bm{#1}}
\newcommand{\Operator}[1]{#1}

% Matrices.
\newcommand{\MA}{\mathcal{A}}
\newcommand{\MB}{\mathcal{B}}
\newcommand{\VF}{\mathcal{F}}

% Gradient and divergence.
\newcommand{\grad}{\Vector{\nabla}}
\newcommand{\diver}{\text{div }}

% Span.
\newcommand{\Span}[1]{\text{span} \left\{ #1 \right\}}

% Bilinear operators.
\newcommand{\boa}[2]{\Operator{a}(#1, #2)}
\newcommand{\bob}[2]{\Operator{b}(#1, #2)}

% Fortin.
\newcommand{\tfortin}{\Operator{\tilde{\Pi}}}
\newcommand{\fortin}{\Operator{\Pi}}
\newcommand{\fortinptpz}{\Operator{\Pi^2_0}}

% Redefinition.
\newcommand{\Exists}{\exists ~}
\newcommand{\Forall}{\forall ~}

% Theorems.
\newtheorem{theorem}{Theorem}[section]
\newtheorem{lemma}{Lemma}[section]
\newtheorem{proposition}{Proposition}[section]

\newtheorem*{theorem*}{Theorem}

\usepackage{courier}
\usepackage{listings}

\lstdefinestyle{default}{
	basicstyle=\ttfamily\color{solarized-base01},
	breakatwhitespace=false,
	breaklines=true,
	keepspaces=true,
	showspaces=false,
	showstringspaces=false,
	showtabs=false,
	tabsize=2
}

% MATLAB, if needed.

% \lstdefinestyle{matlab}{
% 	commentstyle=\color{solarized-green},
% 	keywordstyle=\color{solarized-blue},
% 	stringstyle=\color{solarized-orange},
% 	basicstyle=\ttfamily\color{solarized-base01},
% 	breakatwhitespace=false,
% 	breaklines=true,
% 	captionpos=b,
% 	keepspaces=true,
% 	showspaces=false,
% 	language=matlab,
% 	showstringspaces=false,
% 	showtabs=false,
% 	tabsize=4
% }

\lstset{style=default}

\usepackage{nameref}
\usepackage{multicol}
\usepackage{titlesec}

\usepackage{enumerate}

\usepackage{graphicx}
\graphicspath{{../gallery/}}

\usepackage{xcolor-solarized}
\color{solarized-base02}

\usepackage[T1]{fontenc}
\usepackage[utf8]{inputenc}

\usepackage[a4paper]{geometry}
\geometry{
	inner=20mm,
	outer=20mm,
	top=30mm,
	bottom=30mm,
	heightrounded,
	marginparwidth=50pt,
	marginparsep=20pt,
	headsep=25pt,
	headheight=30pt
}

\usepackage{hyperref}
\hypersetup{
	linktocpage=true,
	colorlinks=true,
	linkcolor=\accentcolor,
	urlcolor=\urlcolor,
	pdftitle={\reporttitle},
	pdfpagemode=FullScreen,
	pdfauthor={Andrea Di Antonio}
}

\usepackage{fancyhdr}
\pagestyle{fancy}
\fancyhf{}
\fancyhead[R]{Andrea Di Antonio}
\fancyhead[L]{\reporttitle}
\fancyfoot[C]{\thepage}

\title{\reporttitle}
\author{Andrea Di Antonio, 858798 \\ \hyperlink{mailto:a.diantonio1@campus.unimib.it}{a.diantonio1@campus.unimib.it}}
\date{Exam session of February 22, 2024 \\ Academic Year 2023-24}

\setcounter{tocdepth}{2}

\begin{document}
	\pagenumbering{roman}
	\maketitle
	\thispagestyle{fancy}

	\begin{abstract}
		\begin{center}
			Report for the course \textit{Advanced Numerical Methods for Partial Differential Equations} focusing on the definition, implementation\footnote{Written in MATLAB.}, and convergence analysis of the Crouzeix-Raviart element for the Stokes Problem.
		\end{center}
	\end{abstract}

	\newpage
	\tableofcontents

	\newpage
	\pagenumbering{arabic}
	\section{Formulations of the Stokes Problem}
	\subsection{Strong formulation}

Consider the domain $\Omega = [0, 1] \times [0, 1] \subset \mathbb{R}^2$. The aim is to find $\Vector{u} \in \left[ C^2(\Omega) \right]^2$ and $p \in C^1(\Omega)$ such that, for any $\Vector{f} \in \left[ C(\Omega) \right]^2$, the following equations hold:

\begin{gather}
    \begin{cases} \label{strong_stokes}
        - \Delta \Vector{u} - \grad p = \Vector{f} & \Forall \Vector{x} \in \Omega, \\
        \diver \Vector{u} = 0 & \Forall \Vector{x} \in \partial \Omega,
    \end{cases}
\end{gather}

where:

\begin{gather}
    \Delta \Vector{u} = \begin{pmatrix}
        u_{1, xx} + u_{1, yy} \\
        u_{2, xx} + u_{2, yy}
    \end{pmatrix} .
\end{gather}

\subsection{Weak formulation}

Now, seeking $\Vector{u} \in \left[ \HO_0(\Omega) \right]^2 = V$ and $p \in L^2_0(\Omega) = Q$ such that, given $\Vector{f} \in V^*$, the following equations are satisfied:

\begin{gather} \label{weak_stokes}
    \begin{cases}
        \boa{\Vector{u}}{\Vector{v}} + \bob{\Vector{v}}{p} = \langle \Vector{f}, \Vector{v} \rangle & \Forall \Vector{v} \in V,\\
        \bob{\Vector{u}}{q} = 0 & \Forall q \in Q,
    \end{cases}
\end{gather}

where $\boa{\cdot}{\cdot}: V \times V \rightarrow \mathbb{R}$ and $\bob{\cdot}{\cdot}: V \times Q \rightarrow \mathbb{R}$ are defined as follows:

\begin{align}
    \boa{\Vector{u}}{\Vector{v}} &= \int_{\Omega} \grad \Vector{u} : \grad \Vector{v} ~ d \omega, \label{a} \\
    \bob{\Vector{v}}{p} &= \int_{\Omega} \diver \Vector{v} ~ p ~ d \omega. \label{b}
\end{align}

Let $Z = \{\Vector{v} \in V: \bob(\Vector{v}, q) = 0 \text{ for all } q \in Q\}$. Consider:

\begin{gather} \label{kernel_stokes}
    \boa{\Vector{u}}{\Vector{v}} = \langle \Vector{f}, \Vector{v} \rangle \quad \Forall \Vector{v} \in V
\end{gather}

This problem is well-posed provided that there exist $\alpha, \beta \in \mathbb{R}_+$ such that:

\begin{gather} \label{coercivity}
    \boa{\Vector{u}}{\Vector{v}} \geq \alpha \lVert \Vector{u} \rVert^2_V \quad \Forall \Vector{u} \in Z,
\end{gather}

which implies existence and unicity for \eqref{kernel_stokes}, and:

\begin{gather} \label{inf-sup}
    \sup_{\Vector{v} \in V} \frac{\bob{\Vector{v}}{q}}{\lVert \Vector{v} \rVert_V} \geq \beta \lVert q \rVert_Q \quad \Forall q \in Q,
\end{gather}

with the latter known as the \textit{inf-sup} property which implies unicity for \eqref{weak_stokes}.

	\newpage
	\section{Finite Element Discretization}
	For the FEM discretization of the Stokes problem, we seek $\Vector{u_h} \in V_h$ and $p_h \in Q_h$ such that the following holds:

\begin{gather} \label{fem_stokes}
    \begin{cases}
        \boa{\Vector{u_h}}{\Vector{v_h}} + \bob{\Vector{v_h}}{p_h} = \langle \Vector{f}, \Vector{v_h} \rangle & \Forall \Vector{v_h} \in V_h, \\
        \bob{\Vector{u_h}}{q_h} = 0 & \Forall q_h \in Q_h.
    \end{cases}
\end{gather}

Considering $Z_h$, \eqref{kernel_stokes} can be rewritten as follows:

\begin{gather} \label{fem_kernel}
    \boa{\Vector{u_h}}{\Vector{v_h}} = \langle \Vector{f}, \Vector{v_h} \rangle \quad \Forall \Vector{v_h} \in V_h.
\end{gather}

We must prove that the bilinear operator $\boa{\cdot}{\cdot}$ is coercive on $Z_h$ as well, considering that we cannot ensure that $Z_h \in Z$. Furthermore, since $V_h$ is smaller than $V$, even \eqref{inf-sup} needs to be revised. 

We observe that, according to \eqref{a}, $\boa{\cdot}{\cdot}$ is coercive on the entirety of $V$ and thus on $Z_h$. We will, however, demonstrate the \textit{inf-sup} property on the Crouzeix-Raviart, as defined later, since it depends on the element.

\subsection{Matrix reformulation}

Let $\left\{ \Vector{\phi_i} \right\}_{i = 1}^N$ and $\left\{ \Vector{\psi_i} \right\}_{i = 1}^M$ denote bases for spaces $V_h$ and $Q_h$ respectively. Then, we express $\Vector{u_h}$ and $p_h$ as:

\begin{align}
    \Vector{u_h} &= \sum_{i = 1}^N \upsilon_i \Vector{\phi_i} \quad \Forall \Vector{u_h} \in V_h, \\
    p_h &= \sum_{i = 1}^M \pi_i \psi_i \quad \Forall p_h \in Q_h,
\end{align}

so that we aim to find $\Vector{\upsilon} \in \R^N$ and $\Vector{\pi} \in \R^M$ such that:

\begin{gather} \label{matrix_stokes}
    \begin{cases}
        \MA \Vector{\upsilon} + \MB \Vector{\pi} = \VF, \\
        \MB^\intercal \Vector{\upsilon} = 0,
    \end{cases}
\end{gather}

where $\MA \in \R^{N \times N}$, $\MB \in \R^{N \times M}$, and $\VF \in \R^N$ are defined as:

\begin{align}
    \MA_{ij} &= \boa{\Vector{\phi_i}}{\Vector{\phi_j}}, \\ 
    \MB_{ij} &= \bob{\Vector{\phi_i}}{\psi_j}, \\
    \VF_i &= \langle \Vector{f}, \Vector{\phi_i} \rangle.
\end{align}

\newpage
\subsection{Discrete \textit{inf-sup} property}

Consider the following result.

\begin{lemma}[Fortin] \label{fortin}
    Let $(V, Q)$ be spaces for the Stokes problem for which we have the continuos \textit{inf-sup} property, and let $(V_h, Q_h)$ be the discrete spaces. Let $\fortin: V \rightarrow V_h$ such that for all $\Vector{v} \in V$:
    \begin{enumerate}[i.]
        \item There exists $C \in \R_+$ such that $\lVert \fortin \Vector{v} \rVert_V \leq C \lVert \Vector{v} \rVert_V$,
        \item $\bob{\fortin \Vector{v}}{q_h} = \bob{\Vector{v}}{q_h}$ for all $q_h \in Q_h$,
    \end{enumerate}
    then we have the discrete \textit{inf-sup} property for $(V_h, Q_h)$.
\end{lemma}

	\newpage
	\section{Crouzeix-Raviart Element}
	\begin{figure}[!ht]
	\centering
	\includegraphics[trim=0cm 8cm 0cm 10cm, clip, width=15cm]{cr.pdf}
	\caption{Crouzeix-Raviart local DOFs for velocity and pressure.}
\end{figure}

Let $\{\Tau_h\}_h$ be a sequence of shape-regular and quasi-uniform meshes. Define $V_h$ and $Q_h$ as follows:

\begin{align}
    V_h &= \left\{ \Vector{v_h} \in H^1_0(\Omega): \Vector{v_h} \vert_T \in \left[ \PK{2}(T) \cup \Span{b_T} \right]^2 ~ \Forall T \in \Tau_h \right\}, \\
    Q_h &= \left\{ q_h \in \LT_0(\Omega): q_h \vert_T \in \PK{1}(T) ~ \Forall T \in \Tau_h \right\},
\end{align}

where $b_T \in \PK{3}(T)$ such that $b_T \vert_{\partial T} = 0$ and $b_T(\nu_T) = 1 ~ \Forall T \in \Tau_h$:

\subsection{Fortin operator}

Let $\tfortin: V \rightarrow V_h$ such that:

\begin{gather}
    \tfortin \Vector{v} \vert_T = \frac{1}{\theta_T} \left[ \int_T \diver \Vector{v} \begin{pmatrix}
        x - x_T \\
        y - y_T
    \end{pmatrix} \right] \Vector{b_T},
\end{gather}

we define the Fortin operator $\fortin: V \rightarrow V_h$ for the Crouzeix-Raviart element as follows:

\begin{gather}
    \fortin \Vector{v} = \fortinptpz \Vector{v} + \tfortin \left( \Vector{v} - \fortinptpz \Vector{v}\right),
\end{gather}

where $\fortinptpz$ is the Fortin operator for the $\PK{2}-\PK{0}$ element.

$\fortin$ satisfies the hypothesis of the \nameref{fortin} lemma, hence ensuring the discrete \textit{inf-sup} property for the Crouzeix-Raviart element.

\newpage
\subsection{Convergence}

We can formulate the following result:

\begin{proposition}
    Let $\{\Tau_h\}_h$ be a sequence of shape-regular and quasi-uniform meshes. Suppose $(\Vector{u}, p)$ is the solution of \eqref{weak_stokes}, and $(\Vector{u_h}, p_h)$ is the solution of \eqref{fem_stokes}. Then, we have that:

    \begin{gather}
        \lVert \Vector{u} - \Vector{u_h} \rVert_V + \lVert p - p_h \rVert_Q \lesssim \inf_{\Vector{v_h} \in V_h} \lVert \Vector{u} - \Vector{v_h} \rVert_V + \inf_{q_h \in Q_h} \lVert p - q_h \rVert_Q \lesssim h^2.
    \end{gather}
\end{proposition}

	\newpage
	\section{Tests}
	Now, the behavior of the $\LT$ and $\HO$ velocity errors, as well as the $\LT$ pressure error\footnote{Linear slope represented in \textcolor{solarized-green}{green}, quadratic slope in \textcolor{solarized-blue}{blue}, and cubic slope in \textcolor{solarized-yellow}{yellow}. Interpolant in \textcolor{solarized-red}{red}.}, on various meshes is being evaluated as the mesh size decreases, testing \ref{convergence}.

\subsection{Error trend}

A cubic trend can be observed for the velocity $\LT$ error, and a quadratic trend can be seen for both its $\HO$ error and pressure $\LT$ error, as expected from \ref{convergence}.

\begin{figure}[!ht]
	\centering
	\includegraphics[width=15cm]{errorTrend.pdf}
	\caption{Error trend for the Crouzeix-Raviart element.}
\end{figure}

From left to right: $\LT$ error on velocity, $\HO$ error on velocity, and $\LT$ error on pressure. Mesh size on the $X$ axis.

\newpage
\subsection{Bubbleless error trend}

By eliminating the cubic bubble from the velocity elements, new error trends can be observed for both velocity and pressure errors, as they decrease with lower speeds.

\begin{figure}[!ht]
	\centering
	\includegraphics[width=15cm]{errorTrendNB.pdf}
	\caption{Error trend for the bubbleless Crouzeix-Raviart element.}
\end{figure}

\newpage
\subsection{Mesh information}

Here is some information on the meshes for the two methods.

\begin{multicols}{2}
	\lstinputlisting{../results/info.txt}
	\lstinputlisting{../results/infoNB.txt}
\end{multicols}

\subsection{Polynomial fits}

Here are the results for the polynomial fits. Given the relatively low pressure error on the initial mesh, an additional fit is performed solely on the tail. It can be observed that there is a quadratic error trend for the Crouzeix-Raviart element and the importance of the bubble as the convergence slows down when it is removed.

\begin{multicols}{2}
	\lstinputlisting{../results/errorTrend.txt}
	\lstinputlisting{../results/errorTrendNB.txt}
\end{multicols}

\end{document}
