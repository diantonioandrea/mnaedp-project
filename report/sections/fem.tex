\subsection{FEM Discretization}

For the FEM discretization of the Stokes problem, the objective is to find $\Vector{u_h} \in V_h$ and $p_h \in Q_h$ such that the following holds:

\begin{gather} \label{fem_stokes}
    \begin{cases}
        \boa{\Vector{u_h}}{\Vector{v_h}} + \bob{\Vector{v_h}}{p_h} = \langle \Vector{f}, \Vector{v_h} \rangle & \Forall \Vector{v_h} \in V_h, \\
        \bob{\Vector{u_h}}{q_h} = 0 & \Forall q_h \in Q_h.
    \end{cases}
\end{gather}

Considering $Z_h$, \eqref{kernel_stokes} can be rewritten as follows:

\begin{gather} \label{fem_kernel}
    \boa{\Vector{u_h}}{\Vector{v_h}} = \langle \Vector{f}, \Vector{v_h} \rangle \quad \Forall \Vector{v_h} \in V_h.
\end{gather}

It is necessary to prove that the bilinear operator $\boa{\cdot}{\cdot}$ is coercive on $Z_h$ as well, considering that $Z_h$ may not be in $Z$. Furthermore, since $V_h$ is smaller than $V$, even \eqref{inf-sup} needs to be revised. 

Observing that, according to \eqref{a}, $\boa{\cdot}{\cdot}$ is coercive on the entirety of $V$ and thus on $Z_h$, it suffices to demonstrate the \textit{inf-sup} property on the Crouzeix-Raviart, as defined later.

\subsection{Matrix Reformulation}

Let $\left\{ \Vector{\phi_i} \right\}_{i = 1}^N$ and $\left\{ \Vector{\psi_i} \right\}_{i = 1}^M$ denote bases for spaces $V_h$ and $Q_h$ respectively. Then, we express $\Vector{u_h}$ and $p_h$ as:

\begin{align}
    \Vector{u_h} &= \sum_{i = 1}^N \upsilon_i \Vector{\phi_i} \quad \Forall \Vector{u_h} \in V_h, \\
    p_h &= \sum_{i = 1}^M \pi_i \psi_i \quad \Forall p_h \in Q_h,
\end{align}

so that we aim to find $\Vector{\upsilon} \in \R^N$ and $\Vector{\pi} \in \R^M$ such that:

\begin{gather} \label{matrix_stokes}
    \begin{cases}
        \MA \Vector{\upsilon} + \MB \Vector{\pi} = \VF, \\
        \MB^\intercal \Vector{\upsilon} = 0,
    \end{cases}
\end{gather}

where $\MA \in \R^{N \times N}$, $\MB \in \R^{N \times M}$, and $\VF \in \R^N$ are defined as:

\begin{align}
    \MA_{ij} &= \boa{\Vector{\phi_i}}{\Vector{\phi_j}}, \\ 
    \MB_{ij} &= \bob{\Vector{\phi_i}}{\psi_j}, \\
    \VF_i &= \langle \Vector{f}, \Vector{\phi_i} \rangle.
\end{align}

\newpage
\subsection{Discrete \textit{inf-sup} Property}

Consider the following result.

\begin{lemma}[Fortin] \label{fortin}
    Let $(V, Q)$ be spaces for the Stokes problem for which the continuous \textit{inf-sup} property holds, and let $(V_h, Q_h)$ be the discrete spaces. Let $\fortin: V \rightarrow V_h$ such that for all $\Vector{v} \in V$:
    \begin{enumerate}[i.]
        \item There exists $C \in \R_+$ such that $\lVert \fortin \Vector{v} \rVert_V \leq C \lVert \Vector{v} \rVert_V$,
        \item $\bob{\fortin \Vector{v}}{q_h} = \bob{\Vector{v}}{q_h}$ for all $q_h \in Q_h$,
    \end{enumerate}
    then the discrete \textit{inf-sup} property holds for $(V_h, Q_h)$.
\end{lemma}

Verifying this property for the Crouzeix-Raviart element is necessary to ensure its well-posedness.