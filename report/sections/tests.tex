Now, the behavior of the $\LT$ and $\HO$ velocity errors, as well as the $\LT$ pressure error\footnote{Linear slope will be represented in \textcolor{solarized-green}{green}, quadratic slope in \textcolor{solarized-blue}{blue}, and cubic slope in \textcolor{solarized-yellow}{yellow}.}, on various meshes is being evaluated as the mesh size decreases, testing \ref{convergence}.

\subsection{Error trend}

A cubic trend can be observed for the velocity $\LT$ error, and a quadratic trend can be seen for both its $\HO$ error and pressure $\LT$ error, as expected from \ref{convergence}.

\begin{figure}[!ht]
	\centering
	\includegraphics[width=15cm]{errorTrend.pdf}
	\caption{Error trend for the Crouzeix-Raviart element.}
\end{figure}

From left to right: $\LT$ error on velocity, $\HO$ error on velocity, and $\LT$ error on pressure. Mesh size on the $X$ axis.

\newpage
\subsection{Bubbleless error trend}

By eliminating the cubic bubble from the velocity elements, new error trends can be observed for both velocity and pressure errors, as they decrease with lower speeds.

\begin{figure}[!ht]
	\centering
	\includegraphics[width=15cm]{errorTrendNB.pdf}
	\caption{Error trend for the bubbleless Crouzeix-Raviart element.}
\end{figure}

\newpage
\subsection{Mesh informations}

Here is some information on the meshes for the two methods.

\begin{multicols}{2}
	\lstinputlisting{../results/info.txt}
	\lstinputlisting{../results/infoNB.txt}
\end{multicols}

\subsection{Polynomial fits}

Here are the results for the polynomial fits. Given the relatively low pressure error on the initial mesh, an additional fit is performed solely on the tail. It can be observed that there is a quadratic error trend for the Crouzeix-Raviart element and the importance of the bubble as the convergence slows down when it is removed.

\begin{multicols}{2}
	\lstinputlisting{../results/errorTrend.txt}
	\lstinputlisting{../results/errorTrendNB.txt}
\end{multicols}