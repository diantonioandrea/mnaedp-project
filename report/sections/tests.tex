Now, the behavior of the $\LT$ and $\HO$ velocity errors, as well as the $\LT$ pressure error\footnote{Linear slope will be represented in \textcolor{solarized-green}{green}, quadratic slope in \textcolor{solarized-blue}{blue}, and cubic slope in \textcolor{solarized-yellow}{yellow}.}, on various meshes is being evaluated as the mesh size decreases, testing \ref{convergence}.

\subsection{Error trend}

A cubic trend can be observed for the velocity $\LT$ error, and a quadratic trend can be seen for both its $\HO$ error and pressure $\LT$ error, as expected from \ref{convergence}.

\begin{figure}[!ht]
	\centering
	\includegraphics[width=15cm]{errorTrend.pdf}
	\caption{Error trend.}
\end{figure}

\newpage
\subsection{Bubbleless error trend}

By eliminating the cubic bubble from the velocity elements, new error trends can be observed for both velocity and pressure errors, as they decrease with lower speeds.

\begin{figure}[!ht]
	\centering
	\includegraphics[width=15cm]{errorTrendNB.pdf}
	\caption{Bubbleless error trend.}
\end{figure}

\newpage
\subsection{Polynomial fits}

Here are the results for the polynomial fits. Given the relatively low pressure error on the initial mesh, an additional fit is performed solely on the tail. 

\lstinputlisting[caption=Polynomial fits.]{../results/errorTrend.txt}
\lstinputlisting[caption=Bubbleless polynomial fits.]{../results/errorTrendNB.txt}